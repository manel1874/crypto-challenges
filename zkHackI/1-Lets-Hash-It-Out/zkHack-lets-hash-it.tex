\documentclass{article}
\usepackage[utf8]{inputenc}
\usepackage{hyperref}
\usepackage{amsmath}
\usepackage{bm}



\title{zkHack | Let's hash it out}
\author{Manuel B. Santos}
\date{December 2022}

\begin{document}

\maketitle

\noindent Puzzle: \url{https://zkhack.dev/events/puzzle1.html}.

\section{Solution}

We are told we have access to $256$ messages $m_1, \ldots, m_{256}$ and their corresponding BLS signatures $\sigma_1, \ldots, \sigma_{256}$.

Recall how these signatures are generated:
$$\sigma_i = sk \cdot H(m_i),$$
where $H(m_i)$ is the Pedersen hash. We can expand the above signature expression by plugging in the definition of $H(m)$. For some general message $m$ and random elements $g_1, \ldots, g_{n}$, 
$$H(m) := \sum_{j=1}^{n} h(m)_j \cdot g_j,$$
where $h$ is some $n-$bit  hash function and $h(m)_j$ is its $j-$th bit. In the context of the challenge, $n=256$ and $h$ is the blake2s hash.

By linearity we have, 
\begin{align}
\sigma_i &= sk \cdot \sum_{j=1}^{256} h(m_i)_j \cdot g_j \nonumber \\
&= \sum_{j=1}^{256} h(m_i)_j \cdot \left(sk \cdot g_j\right) \nonumber \\
&= \sum_{j=1}^{256} h(m_i)_j \cdot pk_j \label{eq:sigs},
\end{align}
where we denote $pk_j = sk\cdot g_j$ for short. 

Therefore, in order to sign some general message $m$, we need to know the elements $pk_j$:

\begin{equation}
\label{eq:main}
    \sigma_m = \sum_{j=1}^{256} h(m)_j \cdot pk_j.
\end{equation}

To find the elements $pk_j$, let us rewrite the expression \eqref{eq:sigs} in matrix format:

\begin{align}
\bm{\sigma} &= M \cdot \bm{pk} \nonumber\\
\begin{pmatrix} \sigma_1 \\ \sigma_2  \\ \vdots  \\ \sigma_{256} \\ \end{pmatrix} &= \begin{pmatrix} h(m_1)_1 & h(m_1)_2 & \dots & h(m_1)_{256} \\ h(m_1)_2 & h(m_2)_2 & \dots & h(m_2)_{256} \\ \vdots & \vdots & \ddots & \vdots \\ h(m_{256})_1 & h(m_{256})_2 & \dots & h(m_{256})_{256} \\ \end{pmatrix} \begin{pmatrix} pk_1 \\ pk_2  \\ \vdots  \\ pk_{256} \\ \end{pmatrix} \nonumber 
\end{align}

So, we have the vector $P$ is given by \begin{equation}
\label{eq:pkeys}
    \bm{pk} = \bm{\sigma} \cdot M^{-1}.
\end{equation}

From expression \eqref{eq:main} and \eqref{eq:pkeys}, the solution is given by:

\begin{center}
\begin{boxed}
{\sigma_m = h(m) \cdot \bm{\sigma} \cdot M^{-1}}
\end{boxed}.
\end{center}







\end{document}
